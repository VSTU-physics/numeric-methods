---
layout: post
title: "Численное решение ОДУ: метод Эйлера"
date:   2014-10-24 22:30:00
categories: ОДУ
---

Пожалуй, самый простой метод решения задачи Коши. Пусть у нас имеется задача
Коши:

$$
    y' = f(x, y),\ y(x_0) = y_0.
$$

Накроем числовую прямую сеткой с шагом $$ h $$ и разложим точное решение вблизи
$$ n $$-го узла:

$$
    y(x) = y(x_n) + y'(x_n)\cdot(x-x_n) + o(x-x_n).
$$

Обозначим $$ y(x_n) = y_n $$ и получим формулу

$$
    y_{n+1} = y_{n} + f(x_n, y_n) \cdot h + o(h)
    \approx y_n + f(x_n, y_n) \cdot h.
$$

Это и есть метод Эйлера. Программная реализация может иметь вид:


import numpy as np
import matplotlib.pyplot as plt

def euler(f, y0, xs):
    ys = [y0]
    for i in range(len(xs) - 1):
        dx = xs[i + 1] - xs[i]
        ys.append(ys[i] + dx * f(xs[i], ys[i]))
    return ys

# Для примера решим парочку простейших уравнений:
# 1. y' = -y, y(0) = 1
xs = np.linspace(0, 3, 31)
ys = euler(lambda x, y: -y, 1, xs)
plt.subplot(211)
plt.title("y' = -y")
plt.plot(xs, ys, "r-", label="метод Эйлера")
plt.plot(xs, np.exp(-xs), "k-", label="точное решение")
plt.legend()

# 2. y' = cos x, y(0) = 0
xs = np.linspace(0, 6.28, 100)
ys = euler(lambda x, y: np.cos(x), 0, xs)
plt.subplot(212)
plt.title("y' = cos x")
plt.plot(xs, ys, "r-", label="метод Эйлера")
plt.plot(xs, np.sin(xs), "k-", label="точное решение")
plt.legend()

plt.show()

