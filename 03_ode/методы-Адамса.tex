---
layout: post
title: "Численное решение ОДУ: методы Адамса"
date:   2015-01-02 22:30:00
categories: ОДУ
---

Методы Адамса основаны на той идее, что $$ y' = f(x,y) $$ можно приблизить многочленом $$ P_k(x) $$. Тогда значение решения на следующем шаге можно найти по формуле

$$
  y_{n+1} = y_n + \int\limits_{x_n}^{x_{n+1}} P_n(x)\,dx.
$$

Для построения интерполирующего многочлена степени $$ k $$ требуется $$ k+1 $$ точек, через которые проходит этот многочлен. Для этого находят первые $$ k + 1 $$ точек каким-либо другим способом, а затем используют метод Адамса.

Исходя из метода интегрирования, основанном на многочлене Лагранжа, можно переписать формулу в виде

$$
  y_{n+1} = y_n + \sum\limits_i c_i f(x_{n-i}, y_{n-i}).
$$

Выделяют 2 вида методов Адамса: экстраполяционные и интерполяционные. В экстраполяционном методе Адамса для нахождения следующей точки используются уже рассчитанные:

$$
  y_{n+1} = y_n + \sum\limits_{i=0}^k c_i f(x_{n-i}, y_{n-i}).
$$

В интерполяционном методе для интерполяции используется искомая точка:

$$
  y_{n+1} = y_n + \sum\limits_{i=0}^k d_i f(x_{n + 1 - i}, y_{n + 1 - i}).
$$

В данном случае мы имеем дело с нелинейным уравнением, для решения которого придётся использовать итерационные методы. Возросшая сложность расчёта здесь позволяет получить более точный результат, чем при использовании экстраполяционного метода.
