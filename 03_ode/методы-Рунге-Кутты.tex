---
layout: post
title: "Численное решение ОДУ: методы Рунге-Кутты"
date:   2014-12-30 22:30:00
categories: ОДУ
---

Рассмотренный ранее метод Эйлера имеет 1 порядок точности. Для достижения большей точности используются более сложные методы, вроде семейства методов Рунге-Кутты, о которых сегодня поговорим.

Снова рассматривается задача Коши

$$
  y' = f(x, y),\quad y(x_0) = y_0,
$$

только теперь связь между значениями в соседних точках имеет вид:

$$
  y(x_{n+1}) = y(x_n) + h\alpha_i k_i,\quad k_i = f(x_n + \beta_i h, y_n + \gamma_{ij}k_j h),
$$

где $$ \alpha_i,\ \beta_i,\ \gamma_{ij} $$ определяются таким образом, чтобы обеспечить необходимую точность. Определить эти параметры не так-то просто, поэтому рассмотрим один из самых распространённых методов этого семейства -- метод 4 порядка точности. Для него параметры выглядят так:

$$
  \alpha_i = \left\{\frac{1}{6}, \frac{1}{3}, \frac{1}{3}, \frac{1}{6} \right\},
  \quad\beta_i = \left\{ 0, \frac{1}{2}, \frac{1}{2}, 1 \right\},
  \quad\gamma_{ij} =
  \begin{pmatrix}
  0 & 0 & 0 & 0 \\
  \frac{1}{2} & 0 & 0 & 0 \\
  0 & \frac{1}{2} & 0 & 0 \\
  0 & 0 & 1 & 0
  \end{pmatrix}.
$$

Обычно его записывают так:

$$
\begin{aligned}
  y_{n+1} & = y_n + \frac{h(k_1 + 2k_2 + 2k_3 + k_4)}{6},\\
  k_1 & = f(x_n, y_n),\\
  k_2 & = f\left(x_n +\frac{h}{2}, y_n + \frac{hk_1}{2}\right),\\
  k_3 & = f\left(x_n +\frac{h}{2}, y_n + \frac{hk_2}{2}\right),\\
  k_4 & = f\left(x_n + h, y_n + k_3\right).
\end{aligned}
$$

Простейший пример кода демонстрирует решение задачи Коши

$$
  y' = x - y,\quad y(0) = 0
$$

на отрезке $$ [0, 1] $$:


from matplotlib import pyplot as plt
from numpy import linspace

def f(x, y):
  return x - y

def main():
  x = linspace(0, 1, 21)
  y = [0]
  h = x[1] - x[0]
  for i in x[1:]:
    j = y[-1]
    k1 = f(i, j)
    k2 = f(i + h / 2, j + k1 * h / 2)
    k3 = f(i + h / 2, j + k2 * h / 2)
    k4 = f(i + h, j + k3 * h)
    y.push(j + (k1 + 2 * k2 + 2 * k3 + k4) * h / 6)
  plt.plot(x, y)

if __name__ == __main__:
  main()

