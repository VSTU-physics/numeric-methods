---
layout: post
title:  "Метод конечных элементов"
date:   2015-01-25 16:00:00
categories: ДУЧП
---

Методы конечных элементов отличаются более продвинутым математическим аппаратом, чем методы конечных разностей. Как следствие, они несколько сложнее в реализации. Но за счёт этой сложности можно достичь лучших результатов.

Будем рассматривать эти методы на примере простейшего уравнения -- стационарного уравнения теплопроводности:

$$
\begin{aligned}
  & -(k(x) u_x')_x' + p(x)u = f(x),\quad x\in[0,1],\\
  & u_x'(0) = u_x'(1) = 0,
\end{aligned}
$$

где

$$
  k(x) \ge k_0 > 0,\quad p(x) > 0.
$$

## Формулировки методов конечных элементов

### Вариационная формулировка (Ритц)

В этой формулировке начальная задача несколько видоизменяется. Вместо дифференциального уравнения рассматривается функционал, построенный так, чтобы решение дифференциального уравнения его минимизировало. В нашем случае этот функционал имеет вид:

$$
  F(u) = (k u_x', u_x') + (p u, u) - 2 (f, u),\quad (u, v) = \int\limits_0^1 u(x) v(x)\,dx.
$$

Нетрудно показать, что точное решение уравнения его минимизирует. Рассмотрим его значение для функции $$ u + \xi $$:

$$
\begin{aligned}
  F(u + \xi) = & (k u_x', u_x') + 2 (k u_x', \xi_x') + (k \xi_x', \xi_x') + (p u, u) + 2 (p u, \xi) + (p \xi, \xi) -\\
  & - 2 (f, u) - 2 (f, \xi) =\\
  & F(u) + (k \xi_x', \xi_x') + (p \xi, \xi) + 2[(ku_x', \xi_x') + (pu, \xi) - (f, \xi)],\\
  & (ku_x', \xi_x') + (pu, \xi) - (f, \xi) = \int\limits_0^1 ku_x'\xi_x'\,dx + \int\limits_0^1 (pu - f)\xi\,dx = \\
  & ku_x'\xi\Big|_0^1 + \int\limits_0^1 (-(ku_x')_x' + pu - f)\xi\,dx = 0.
\end{aligned}
$$

Для того, чтобы найти решение уравнения, разложим его по некоторому базису $$ \{ \psi_k \} $$:

$$
  u = \sum\limits_{k=1}^\infty c_k \psi_k.
$$

Проблема здесь в том, что задачу с бесконечным числом коэффициентов придётся решать бесконечно долго. Поэтому стараются выбирать базис так, чтобы коэффициенты достаточно быстро убывали и ограничиваются некоторым числом $$ N $$:

$$
  u \approx \sum\limits_{k=1}^N c_k \psi_k.
$$

Тогда функционал становися функцией $$ N $$ переменных:

$$
  F(u) = F(c_1,\ldots,c_N),
$$

а для его минимизации можно воспользоваться необходимым условием экстремума функции многих переменных:

$$
  F_{c_k}' = 0, \quad k = \overline{1,N}.
$$

### Проекционная формулировка (Галёркин)

Возможен и другой подход к определению коэффициентов разложения. Снова разложим искомую функцию по некоторому конечному функциональному базису:

$$
  u \approx \sum\limits_{k=1}^N c_k \psi_k.
$$

Невязку $$ R $$ определим, используя уравнение:

$$
  R(x) = -\left[k(x) \sum\limits_{k=1}^N(\psi_k)_x'\right]_x' +
  p(x)\sum\limits_{k=1}^N\psi_k - f(x).
$$

Естественным желанием будет минимизировать невязку. Из требования
$$ \|R\| \to \min $$ следует

$$
  (R, \psi_l) = 0,\quad l = \overline{1,N},
$$

то есть получается система уравнений:

$$
  \sum\limits_{m=1}^N c_m \left[-(k\psi_m'', \psi_l) + (k'\psi_m', \psi_l) +
  (p\psi_m, \psi_l)\right] = -(f, \psi_l).
$$

## Уравнение теплопроводности
