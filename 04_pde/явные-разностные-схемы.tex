---
layout: post
title:  "Явные разностные схемы"
date:   2014-11-06 16:00:00
categories: ДУЧП
---

В этой статье я затрону явные способы решения параболических и гиперболических
уравнений.

## Параболические уравнения

Рассмотрим параболическое уравнение в нормальных координатах:

$$
    \frac{\partial u}{\partial t} = \frac{\partial^2 u}{\partial x^2} + f(x, t),
    \quad  x\in(0,1),\ t > 0.
$$

Начальное и граничные условия в общем виде выглядят так:

$$
  u(x, 0) = u_0(x),\quad
  \left.\left(a_0 \frac{\partial u}{\partial x} + b_0 u\right)\right|_{x=0} = c_0(t),
  \quad
  \left.\left(a_1 \frac{\partial u}{\partial x} + b_1 u\right)\right|_{x=1} = c_1(t).
$$

Для численного решения построим сетку с шагом $$ \xi $$ по оси $$ x $$
и $$ \tau $$ по оси $$ t $$ и заменим дифференциальные операторы разностными:

$$
    \frac{u_{i, j+1} - u_{i, j}}{\tau} =
    \frac{u_{i+1, j} - 2u_{i, j} + u_{i-1, j}}{\xi^2} + f_{i, j}.
$$

Отсюда можно выразить значение на следующем шаге по времени через предыдущий:

$$
    u_{i, j+1} - u_{i, j} = r(u_{i+1, j} - 2u_{i, j} + u_{i-1, j}) +
    \tau\cdot f_{i, j},\quad r = \frac{\tau}{\xi^2},
$$

$$
    u_{i, j+1} = (1 - 2r)u_{i, j} + r(u_{i+1, j} + u_{i-1, j})
    + \tau\cdot f_{i, j},\quad i \in \{1,\ldots, n-1\}.
$$

При этом начальное условие позволит определить начальные значения:

$$
    u_{i, 0} = u_0(i \cdot \xi),
$$

а граничные -- значения на краях ($$ i \in \{0, n\} $$):

$$
  a_0 \frac{u_{1, j} - u_{0, j}}{\xi} + b_0 u_{0, j} = c_0(j\cdot\tau),\quad
  a_1 \frac{u_{n, j} - u_{n-1, j}}{\xi} + b_1 u_{n, j} = c_1(j\cdot\tau).
$$

Рассмотрим работу этой схемы на примере задачи о распределении температур в
стержне.


from matplotlib import pyplot as plt
from numpy import sin, exp, pi


def solve_parabolic(start, border_left, border_right, source, ts, te):
    """
    start -- начальное условие
    border_*(t) -- граничные
    source(x, t) -- функция источника
    ts -- время начала эксперимента
    te -- время конца эксперимента
    """
    u = list(start)

    # количество шагов на оси x
    n = len(u) - 1
    xi = 1.0 / n

    # шаг по времени (из условия Куранта)
    r = 0.45 # < 0.5
    tau = r * xi ** 2

    # копия, пригдится при расчёте
    v = list(u)

    # текущее время
    t = ts

    while t < te:
        # пересчёт значений в середине по разносной схеме
        for i in range(1, n):
            v[i] = (1 - 2 * r) * u[i] +\
                    r * (u[i-1] + u[i+1]) +\
                    tau * source(i * xi, t)
        # пересчёт значений на краях
        a0, b0, c0 = border_left(t)
        a1, b1, c1 = border_right(t)
        v[0] = (c0 - v[1] * a0 / xi) / (b0 - a0 / xi)
        v[n] = (c1 + v[n-1] * a1 / xi) / (b1 + a1 / xi)
        # нет времени объяснять
        u, v = v, u
        # обновление времени
        t += tau

    return u

# Рассмотрим решение на примере задачи о распределении температур в стержне
# Пусть у достаточно тонкого стержня отсутствуют тепловые потери через боковую
# поверхность. При этом левый его конец приведён в тепловой контакт с
# термостатом, а правый -- теплоизолирован. Внутри стержня посередине находится
# достаточно маленький источник тепла. Требуется определить распределение
# температур в стержне

n = 100

# задаём начальное условие
u = [0.0 for i in range(n + 1)]

# граничные
locked = lambda t: [0.0, 1.0, 0.0]           # контакт с термостатом
free =   lambda t: [1.0, 0.0, 0.0]           # теплоизолированный конец
source = lambda x, t: 1 if abs(x-0.5) < 1.0 / n else 0  # "точечный" источник посередине

X = [1.0 / n * i for i in range(n+1)]

# профиль функции
plt.plot(X, u)

dt = 0.1
for i in range(7):
    u = solve_parabolic(u, locked, free, source, i * dt, (i + 1) * dt)
    plt.plot(X, u)

# покажи мне это
plt.show()




## Гиперболическое уравнение

Гиперболическое уравнение имеет вид

$$
    \frac{\partial^2 u}{\partial t^2} =
    \frac{\partial^2 u}{\partial x^2} + f(x, t),
    \quad  x\in(0,1),\ t > 0.
$$

Начальных условий у него пара:

$$
  u(x, 0) = u_0(x),\quad u'_t(x, 0) = u_1(x),
$$

а граничные можно записать в том же виде, что и для параболического:

$$
  \left.\left(a_0 \frac{\partial u}{\partial x} + b_0 u\right)\right|_{x=0} = c_0(t),
  \quad
  \left.\left(a_1 \frac{\partial u}{\partial x} + b_1 u\right)\right|_{x=1} = c_1(t).
$$

Заменяя дифференциальные операторы разностными, приходим к разностной схеме:

$$
    \frac{u_{i, j+1} - 2u_{i, j} + u_{i,j-1}}{\tau^2} =
    \frac{u_{i+1, j} - 2u_{i, j} + u_{i-1, j}}{\xi^2} + f_{i, j}.
$$

Отсюда можно выразить значение на следующем шаге по времени через 2 предыдущих:

$$
    u_{i, j + 1} - 2u_{i, j} +  u_{i, j - 1} =
    r(u_{i+1, j} - 2u_{i, j} + u_{i-1, j}) +
    \tau^2\cdot f_{i, j},\quad r = \frac{\tau^2}{\xi^2},
$$

$$
    u_{i, j+1} = 2(1 - r)u_{i, j} + r(u_{i+1, j} + u_{i-1, j}) - u_{i, j-1}
    + \tau^2\cdot f_{i, j},\quad i \in \{1,\ldots, n-1\}.
$$

Следовательно, для запуска схемы нужны значения на 2 шагах. Для их получения
воспользуемся начальными условиями:

$$
    u_{i, 0} = u_0(i \cdot \xi),\quad
    u_{i, 1} = u_{i, 0} + u_1(i \cdot \xi) \cdot \tau.
$$

Со значениями на краях ситуация аналогична параболической схеме:

$$
  a_0 \frac{u_{1, j} - u_{0, j}}{\xi} + b_0 u_{0, j} = c_0(j\cdot\tau),\quad
  a_1 \frac{u_{n, j} - u_{n-1, j}}{\xi} + b_1 u_{n, j} = c_1(j\cdot\tau).
$$

Эту схему можно использовать для решения задачи о колебаниях струны:


from matplotlib import pyplot as plt
from numpy import sin, exp, pi


def solve_hyperbolic(start, der, border_left, border_right, source, ts, te):
    """
    start -- начальное условие
    der -- начальное условие на производную
    border_*(t) -- граничные
    source(x, t) -- функция источника
    ts -- время начала эксперимента
    te -- время конца эксперимента
    """
    u = list(start)

    # количество шагов на оси x
    n = len(u) - 1
    xi = 1.0 / n

    # шаг по времени (из условия Куранта)
    r = 0.95 # < 1
    tau = xi * r ** 0.5

    # копии, пригдятся при расчёте
    v = list(u)
    w = list(u)

    # следующий шаг из начального условия на производную
    v = [ui + tau * der[i] for i, ui in enumerate(u)]

    # текущее время
    t = ts

    while t < te:
        # пересчёт значений в середине по разносной схеме
        for i in range(1, n):
            w[i] = 2 * (1 - r) * v[i] +\
                    r * (v[i-1] + v[i+1]) -\
                    u[i] + tau * tau * source(i * xi, t)
        # пересчёт значений на краях
        a0, b0, c0 = border_left(t)
        a1, b1, c1 = border_right(t)
        v[0] = (c0 - v[1] * a0 / xi) / (b0 - a0 / xi)
        v[n] = (c1 + v[n-1] * a1 / xi) / (b1 + a1 / xi)
        # нет времени объяснять
        u, v, w = v, w, u
        # обновление времени
        t += tau

    d = [(v[i] - u[i]) / tau for i in range(n + 1)]
    return u, d

#

n = 100

# задаём начальное условие
u = [sin(pi * i / n) for i in range(n + 1)]
der = [0] * (n+1)

# граничные
locked = lambda t: [0.0, 1.0, 0.0]  # закреплённые концы
source = lambda x, t: 0

X = [1.0 / n * i for i in range(n+1)]

# профиль функции
plt.plot(X, u)

dt = 0.1
for i in range(10):
    u, der = solve_hyperbolic(u, der, locked, locked, source, i*dt, (i+1)*dt)
    plt.plot(X, u)

# покажи мне это
plt.show()

# в следующий раз я научу вас делать анимацию


