---
layout: post
title:  "Численное интегрирование: многочлен Лагранжа"
date:   2014-09-18 15:00:00
categories: интегрирование
---

Рассмотрим многочлен вида

$$
    P_n^0(x) = \frac{x - x_1}{x_0 - x_1} \cdot \frac{x - x_2}{x_0 - x_2} \cdot
    \cdots \cdot \frac{x - x_n}{x_0 - x_n} =
    \prod_{i=1}^n \frac{x - x_i}{x_0 - x_i}.
$$

Очевидно, что $$ P_n^0(x_i) = \delta_{0i} $$. Сложив $$ n+1 $$ подобных
многочленов можно получить

$$
    L_n(x) = \sum_{i=0}^n f(x_i) \prod_{j=0}^n' \frac{x - x_j}{x_i - x_j},
$$

где штрих у произведения означает, что произведение берётся по всем
$$ j \neq i $$. Этот многочлен называется многочленом Лагранжа. Нетрудно видеть,
что

$$
    L_n(x_j) = \sum_{i=0}^n f(x_i) \delta_{ij} = f(x_j),
$$

то есть этот многочлен совпадает с интерполяционным $$ P_n(x) $$. Однако, при
его интегрировании обнаруживается интересная особенность: для интегрирования
нужно знать лишь интегралы от полиномов вроде рассмотренного выше
$$ P_n^0(x) $$:

$$
    \int\limits_a^b L_n(x)\,dx =
    \int\limits_a^b
    \sum_{i=0}^n f(x_i) \prod_{j=0}^n' \frac{x - x_j}{x_i - x_j}\,dx =
    \sum_{i=0}^n f(x_i)
    \int\limits_a^b\prod_{j=0}^n' \frac{x - x_j}{x_i - x_j}\,dx =
    \sum_{i=0}^n c_i f(x_i),
$$

где коэффициенты

$$
    c_i = \int\limits_a^b\prod_{j=0}^n' \frac{x - x_j}{x_i - x_j}\,dx
$$

зависят только от разбиения и не зависят от интегрируемой функции.
Следовательно, они общие для всех функций и для расчёта интеграла больше
не нужно решать систему.

Для большего обобщения можно рассмотреть отображение

$$
    y : [a, b] \to [-1, 1],\ y(x) = \frac{2x - b - a}{b - a}
$$

Тогда

$$
    c_i = \frac{b - a}{2}\int\limits_{-1}^{1}
          \prod_{j=0}^n' \frac{y - y_j}{y_i - y_j}\,dy.
$$

Точки $$ y_i $$ называются узлами разбиения, а соответствующие им значения
$$ c_i $$ -- весами.

Теперь обратимся к задаче поиска весов. Подынтегральное произведение является многочленом
степени $$ n $$ и может быть представлено в виде

$$
    \prod_{j=0}^n' \frac{y - y_j}{y_i - y_j} = \frac{P_n(y)}{P_n(y_i)}.
$$

Для определения коэффициентов $$ P_n(y) $$ рассмотрим вспомогательную задачу:
определим коэффициенты произведения многочлена степени $$ n $$ на двучлен
$$ (x - a) $$. Расписывая, получим:

$$
    P_n(x)\cdot(x-a) = x P_n(x) - a P_n(x) = \sum_{i=1}^{n+1} p_{i-1} x^i -
    \sum_{i=0}^n ap_i x^i = p_n x^{n + 1} + \sum_{i=1}^n (p_{i-1} - ap_i) x^i -
    ap_0.
$$

Этот процесс можно описать в программе, представив многочлен в виде списка его
коэффициентов:


def polynomial_multiply(p, a):
    '''
        Вычисляет коэффициенты многочлена, который получается при умножении
        многочлена с коэффициентами p на (x-a)

        p -- список коэффициентов [a0, a1, … , an]
    '''
    n = len(p)
    result = []

    # начинаем формировать результат:
    # 1. добавляем свободный член
    result.append(-a * p[0])

    # 2. добавляем все коэффициенты, кроме самого старшего
    for i in range(1, n):
        result.append(p[i-1] - a * p[i])

    # 3. добавляем старший коэффициент
    result.append(p[-1])
    return result

# пример использования:
polynomial_multiply([1], 2)     # [-2, 1]    : 1 * (x - 2) = x - 2
polynomial_multiply([-2, 1], 2) # [4, -4, 1] : (x - 2) * (x - 2) = x^2 - 4x + 4


Используя полученное выражение для коэффициентов, можно определить коэффициенты
$$ P_n(y) $$, последовательно домножая $$ 1 $$ на  $$ y - y_j\ (j \ne i) $$:


def polynomial_from_roots(ys):
    '''
        Вычисляет список коэффициентов многочлена, имеющего корни ys
    '''
    result = [1]
    for y in ys:
        result = polynomial_multiply(result, y)
    return result


Зная коэффициенты, можно рассчитать значение многочлена в любой точке. Это нам
потребуется для определения знаменателя:


def polynomial_value(p, x):
    '''
        Вычисляет значение многочлена с коэффициентами p в точке x
    '''
    return sum(a * x ** i for i, a in enumerate(p))


Для определения веса нужно проинтегрировать полученный многочлен на промежутке
$$ [-1, 1] $$:

$$
    \int\limits_{-1}^1 P_n(x)\,dx = \int\limits_{-1}^1\sum_{i=0}^n a_i x^i\,dx
    = \sum_{i=0}^n a_i \int\limits_{-1}^1 x^i\,dx =
    \sum_{i=0}^n a_i \frac{1 + (-1)^i}{i+1}.
$$

Видно, что слагаемые с нечётными $$ i $$ обращаются в $$ 0 $$, что позволяет
написать вот такой вот код:


def polynomial_integrate(p):
    '''
        Интегрирует многочлен с коэффициентами p на [-1, 1]
    '''
    return sum(2 * a / (i + 1) for i, a in enumerate(p) if i % 2 == 0)


Ну а теперь, собрав всё вместе, можно определить веса:


def weights(ys):
    '''
        Рассчитывает веса
    '''
    result = []
    for i, y in enumerate(ys):
        # сначала формируем список [yj], j ≠ i
        yj = ys[:i] + ys[i + 1:]

        # определяем коэффициенты числителя
        p = polynomial_from_roots(yj)

        # определяем знаменатель
        value = polynomial_value(p, y)

        # рассчитываем вес и добавляем его в список весов
        result.append(polynomial_integrate(p) / value / 2)
    return result


> Внимательный читатель заметит, что произведения отличаются друг от друга
> только одним множителем, поэтому их можно не пересчитывать несколько раз.
> Вместо этого можно посчитать произведение всех двучленов, а затем получать
> каждый многочлен после однократного деления полного произведения на
> $$ y - y_i $$. Для этого можно воспользоваться схемой Горнера.

Ну а теперь можно коротко и ясно описать интегрирование:


def integrate_lagrange(f, a, b, n):
    '''
        Интегрирование полиномом Лагранжа степени n
    '''
    # создаём разбиение отрезка [-1, 1]
    ys = [-1 + 2 * float(i) / n for i in range(n + 1)]

    # считаем веса:
    w = weights(ys)

    # считаем ответ
    x = lambda y: (b + a) / 2 + y * (b - a) / 2 # отображает разбиение ys на
                                                # промежуток интегрирования
    s = sum(w[i] * f(x(y)) for i, y in enumerate(ys))
    return s * (b - a)


> Обратите внимание, что пределы интегрирования нужно передавать как числа с
> плавающей точкой, то есть не 0 или 1, а 0.0 и 1.0. Это особенность python2.
