---
layout: post
title:  "Численное интегрирование"
date:   2014-09-04 12:00:00
categories: интегрирование
---

Существует несколько принципиально различных семейств методов численного
интегрирования функций:

1. методы Монте-Карло;
2. аппроксимация функцией с известным значением интеграла:
  + [интерполирующим полиномом (через вычисление его коэффициентов)]({{ "/интегрирование/интерполирующий_многочлен/" | prepend: site.baseurl }}),
  + [многочленом Лагранжа (через вычисление весов)]({{ "/интегрирование/многочлен_Лагранжа/" | prepend: site.baseurl }});
  + [квадратурные формулы Гаусса]({{ "/интегрирование/формулы_Гаусса/" | prepend: site.baseurl }});
  + [адаптивные методы]({{ "/интегрирование/адаптивные_методы/" | prepend: site.baseurl }});
 (метод прямоугольников, трапеций, парабол)
3. [tanh-sinh, erf, sinh-sinh и им подобные квадратуры, основанные на замене
   переменной и формуле Эйлера-Маклорена]({{ "/интегрирование/tanh-sinh/" | prepend: site.baseurl }});
4. какие-нибудь ещё.



