---
layout: post
title:  "Численное интегрирование: адаптивные методы"
date:   2014-09-18 15:00:00
categories: интегрирование
---

Ранее я описывал методы численного интегрирования, которые сводились
к приближению искомой функции многочленом и интегрированию этого многочлена.
Точность в них увеличивалась засчёт увеличения числа узлов интерполяции.
Но это не единственный способ увеличить точность.

Другим подходом к увеличению точности выступает разбиение промежутка
интегрирования на меньшие частии применение квадратур уже к ним. Замечено, что
чем меньше длина шага интегрирования -- одного такого подпромежутка -- тем выше
точность. Поэтому уже не обязательно использовать большое число точек. Наиболее
употребительными являются методы прямоугольников, трапеций и парабол.

## Метод прямоугольников

Суть его заключается в следующем: на каждом отрезке подынтегральная функция
$$ f(x) $$ приближается функцией $$ f_i(x) = f(x_i) = const $$, где точка
$$ x_i $$ -- внутренняя точка $$ i $$-го промежутка. В зависимости от выбора
этой точки различают:

* левые прямоугольники, когда она совпадает с левым концом промежутка
* центральные прямоугольники, когда она делит промежуток пополам
* правые прямоугольники, когда она самая правая

Формула простая до безобразия:

$$
    I = \int\limits_a^b f(x)\,dx \approx h\sum_i f(x_i),
$$

где $$ h $$ -- длина шага.

## Метод трапеций

Всё то же самое, только

$$
    f_i(x) = k_ix + b_i,\ f_i(x_i) = f(x_i),\ f_i(x_{i+1}) = f(x_{i+1}),
$$

где $$ x_i $$ -- левые концы интервалов. Для значения интеграла получаем

$$
    I = \int\limits_a^b f(x)\,dx \approx \frac{h}{2}\sum_i [f(x_i) + f(x_{i+1})]
      = h\sum_{i=1}^{n-1} + \frac{h}{2}[f(x_0) + f(x_n)].
$$

## Метод парабол

$$
    f_i(x) = a_ix^2 + b_ix + c_i,
    \ f_i(x_i) = f(x_i),
    \ f_i(x_i + h/2) = f(x_i + h/2),
    \ f_i(x_{i+1}) = f(x_{i+1}),
$$

$$
    I = \int\limits_a^b f(x)\,dx \approx
    \frac{h}{6}\sum_i [f(x_i) + 4f(x_i + h/2) + f(x_{i+1})].
$$

## Есть мнение…

Логичнее использовать на каждом шаге формулу Гаусса, так как она точнее,
а веса и корни можно посчитать один раз и дальше просто искать суммы.

